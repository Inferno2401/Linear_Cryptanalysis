\documentclass{article}
\usepackage{geometry}
\usepackage{enumitem}
\usepackage{amsmath}
\usepackage{amssymb}
\usepackage{amsthm}
\usepackage{hyperref}
\geometry{top = 0.5 in, right = 1 in , left = 1 in}
\hypersetup{colorlinks=true}

\begin{document}

    \title{\centering{Project Report} \\ \large{CS406 - Cryptography}}
    \author{Ameya Vikrama Singh (210070007) \\ Krishna Agaram (210051003)}
    \maketitle
    \section{Implementing DES}
    \subsection{Feistel Network}
    A Feistel network is an invertible Pseudo-Random function construction. It is used in implementing block ciphers from the heuristic of Substitution-Permutation Ciphers. A Feistel network consists of multiple "rounds". In each round, we do the following:
    $$F_n(a_n, b_n) = (b_n, a_n \oplus f(b_n))$$

    where $f$ is a (supposedly) pseudorandom function henceforth called the Feistel function. The Feistel construction is provably pseudorandom under the assumption that $f$ is pseudorandom.

    \subsection{DES - The Data Encryption Standard}
    The Data Encryption Standard uses a Feistel network of 16 rounds.
    


\end{document}